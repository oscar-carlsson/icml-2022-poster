
\DeclareCaptionFormat{custom}
{%
    %\textbf{#1#2}\textit{\small #3}
    \rmfamily{\small{\color{dblue}#1#2\color{black} #3}}
}
\captionsetup{format=custom}

\newtheorem*{namedtheorem}{\theoremname}
\newcommand{\theoremname}{testing}
\newenvironment{named}[1]{\renewcommand\theoremname{#1}
    \begin{namedtheorem}}
    {\end{namedtheorem}}

%\newtheorem{theorem}{Theorem}[section]
%\newtheorem{lemma}[theorem]{Lemma}
%\newtheorem{proposition}[theorem]{Proposition}
%\newtheorem{corollary}[theorem]{Corollary}
%\newtheorem{conjecture}[theorem]{Conjecture}
%\newtheorem{claim}[theorem]{Claim}
%
%\theoremstyle{definition}
%\newtheorem{definition}[theorem]{Definition}
%\newtheorem{construction}[theorem]{Construction}
%\newtheorem{discussion}[theorem]{Discussion}
%\newtheorem{assumption}[theorem]{Assumption}
%\newtheorem{assumptions}[theorem]{Assumptions}
%\newtheorem{convention}[theorem]{Convention}
%\newtheorem{example}[theorem]{Example}
%\newtheorem{examples}[theorem]{Examples}
%\newtheorem*{acknowledgement}{Acknowledgement}

%\theoremstyle{remark}
%\newtheorem{remark}[theorem]{Remark}
%\newtheorem{remarks}[theorem]{Remarks}
%\newtheorem{notation}[theorem]{Notation}
%
%\numberwithin{equation}{section}
%\setbeamertemplate{caption}[numbered]

% \numberwithin{figure}{section}
% \renewcommand{\thefigure}{\thesection.\arabic{figure}}



%%%%%%%%%%%%%%%%%%%%%%%%% New math commands %%%%%%%%%%%%%%%%%%%%%%%%%
\DeclareMathOperator{\down}{down}
\DeclareMathOperator{\up}{up}
\DeclareMathOperator{\conv}{conv}
\DeclareMathOperator{\maxPool}{maxPool}
\DeclareMathOperator{\tConv}{tConv}
\DeclareMathOperator{\upsample}{upsample}
\DeclareMathOperator{\StwoSOthreeconv}{S2SO3conv}
\DeclareMathOperator{\SOthreeconv}{SO3conv}
\DeclareMathOperator{\SOthreeStwoconv}{SO3S2conv}

\DeclareMathOperator{\vrize}{vec}

% diagonal matrix
\DeclareMathOperator{\diag}{diag}

% identity
\DeclareMathOperator{\id}{id}

% argmax
\DeclareMathOperator*{\argmax}{argmax}

\newcommand*\IsoTo{%
    \xrightarrow[]{\raisebox{-0.5 em}{\smash{\ensuremath{\sim}}}}%
}

\newcommand{\mf}[1]{\mathfrak{#1}}
\newcommand{\rad}{\mathrm{Rad}\,}
\newcommand{\GL}{\mathrm{GL}}
\newcommand{\SO}{\mathrm{SO}}
\newcommand{\mO}{\mathrm{O}}
\newcommand{\SE}{\mathrm{SE}}
\newcommand{\E}{\mathrm{E}}
\newcommand{\RR}{\mathbb{R}}
\newcommand{\ZZ}{\mathbb{Z}}
\newcommand{\Nc}[1][]{N_{#1}}
\newcommand{\um}{\id}  % unit matrix
\newcommand{\vol}{\mathrm{vol}}

\DeclareMathOperator{\Hom}{Hom}



% this style is applied by default to any tikzpicture included via \tikzfig
\tikzstyle{tikzfig}=[baseline=-0.25em,scale=0.5]

% these are dummy properties used by TikZiT, but ignored by LaTex
\pgfkeys{/tikz/tikzit fill/.initial=0}
\pgfkeys{/tikz/tikzit draw/.initial=0}
\pgfkeys{/tikz/tikzit shape/.initial=0}
\pgfkeys{/tikz/tikzit category/.initial=0}

% standard layers used in .tikz files
\pgfdeclarelayer{edgelayer}
\pgfdeclarelayer{nodelayer}
\pgfsetlayers{background,edgelayer,nodelayer,main}

% Node styles
\tikzstyle{red_dot}=[fill=red, draw=black, shape=circle]
\tikzstyle{green_dot}=[fill=green, draw=black, shape=circle]

% Edge styles
\tikzstyle{right_arrow}=[->]
\tikzstyle{left_arrow}=[<-]
\tikzstyle{thick_right_arrow}=[->, thick]
\tikzstyle{thick_left_arrow}=[<-, thick]
\tikzstyle{filled_path_edge}=[-, fill={rgb,255: red,209; green,209; blue,209}]

% style for blank nodes
\tikzstyle{none}=[inner sep=0mm]

\newcommand{\yrcite}[1]{\citeyearpar{#1}}
\renewcommand{\cite}[1]{\citep{#1}}
% modification to natbib citations
\setcitestyle{authoryear,round,citesep={;},aysep={,},yysep={;}}

%Tool for drawing in tikz
% Used as “\draw <lower left> to[grid with coordinates,ticks on top=true,ticks right=true] <top right>;”
\makeatletter
\def\grd@save@target#1{%
    \def\grd@target{#1}}
\def\grd@save@start#1{%
    \def\grd@start{#1}}
\def\GridCore{\edef\grd@@target{(\tikzinputsegmentlast)}%
    \tikz@scan@one@point\grd@save@target\grd@@target\relax
    \edef\grd@@start{(\tikzinputsegmentfirst)}%
    \tikz@scan@one@point\grd@save@start\grd@@start\relax
    \draw[minor help lines] (\tikzinputsegmentfirst) grid (\tikzinputsegmentlast);
    \draw[major help lines] (\tikzinputsegmentfirst) grid (\tikzinputsegmentlast);
    \grd@start
    \pgfmathsetmacro{\grd@xa}{\the\pgf@x/1cm}
    \pgfmathsetmacro{\grd@ya}{\the\pgf@y/1cm}
    \grd@target
    \pgfmathsetmacro{\grd@xb}{\the\pgf@x/1cm}
    \pgfmathsetmacro{\grd@yb}{\the\pgf@y/1cm}
    \pgfmathsetmacro{\grd@xc}{\grd@xa + \pgfkeysvalueof{/tikz/grid with coordinates/major step}}
    \pgfmathsetmacro{\grd@yc}{\grd@ya + \pgfkeysvalueof{/tikz/grid with coordinates/major step}}
    \foreach \x in {\grd@xa,\grd@xc,...,\grd@xb}
    {\ifticksB
        \node[anchor=north] at (\x,\grd@ya) {\pgfmathprintnumber{\x}};
        \fi
        \ifticksT
        \node[anchor=south] at (\x,\grd@yb) {\pgfmathprintnumber{\x}};
        \fi
    }
    \foreach \y in {\grd@ya,\grd@yc,...,\grd@yb}
    {\ifticksL
        \node[anchor=east] at (\grd@xa,\y) {\pgfmathprintnumber{\y}};
        \fi
        \ifticksR
        \node[anchor=west] at (\grd@xb,\y) {\pgfmathprintnumber{\y}};
        \fi}
}
\newif\ifticksL
\newif\ifticksR
\newif\ifticksT
\newif\ifticksB
\tikzset{ticks left/.is if=ticksL,
    ticks right/.is if=ticksR,
    ticks on top/.is if=ticksT,
    ticks at bottom/.is if=ticksB,
    ticks left=true,
    ticks at bottom=true,
    ticks right=false,
    ticks on top=false,
    grid with coordinates/.style={
        decorate,decoration={show path construction,
            lineto code={\GridCore
        }}
    },
    minor help lines/.style={
        help lines,
        step=\pgfkeysvalueof{/tikz/grid with coordinates/minor step}
    },
    major help lines/.style={
        help lines,
        line width=\pgfkeysvalueof{/tikz/grid with coordinates/major line width},
        step=\pgfkeysvalueof{/tikz/grid with coordinates/major step}
    },
    grid with coordinates/.cd,
    minor step/.initial=.2,
    major step/.initial=1,
    major line width/.initial=2pt,
}
\makeatother

% Command to make rulesep work nicely with subcaptions
\newcommand{\rulesep}{\unskip\ \vrule height -1ex\ }

%%%%%%%%%%%%%%%%%%%%%%%%%%%%%%%% Settings for personal and general TODO notes %%%%%%%%%%%%%%%%%%%%%%%%%%%%%%%
%
%\setlength{\marginparwidth}{1.5cm}  % Necessary to make the todonotes fit into the margins
%
%\newcounter{todocounter}
%
%\colorlet{jgcolor}{green!40!white}
%\newcommand{\jgnote}[1]{\refstepcounter{todocounter}\todo[color=jgcolor,linecolor=black,size=\scriptsize,caption={\textbf{\thetodocounter. JG} #1}]{\textbf{\thetodocounter. JG:}\\#1}{}}
%\newcommand{\jginline}[2][]{
%    \ifthenelse { \equal {#1} {} }
%    { \def\temp {#2} }  % if #1 == blank
%    { \def\temp {#1} }   % else (not blank)
%    \refstepcounter{todocounter}\todo[color=jgcolor,inline,caption={\textbf{\thetodocounter. JG} \temp}]{\textbf{\thetodocounter. JG:} #2}{}}
%\newcommand{\jgblock}[2][{}]{
%    \ifthenelse { \equal {#1} {} }
%    { \def\templist {\emph{block comment}}
%        \def\tempheader {}}  % if #1 == blank
%    { \def\templist {#1}
%        \def\tempheader {#1}}   % else (not blank)
%    \refstepcounter{todocounter}\todo[color=jgcolor,inline,caption={\textbf{\thetodocounter. JG} \templist}]{\textbf{\thetodocounter. JG: \tempheader}\\\begin{minipage}{\textwidth}#2\end{minipage}}{}}
%\newcommand{\jgfigure}[1]{\refstepcounter{todocounter}\missingfigure[figcolor=jgcolor]{\textbf{\thetodocounter. JG:} #1}{}}
%
%
%\colorlet{jacolor}{blue!20!white}
%\newcommand{\janote}[1]{\refstepcounter{todocounter}\todo[color=jacolor,linecolor=black,size=\scriptsize,caption={\textbf{\thetodocounter. JA} #1}]{\textbf{\thetodocounter. JA:}\\#1}{}}
%\newcommand{\jainline}[2][]{
%    \ifthenelse { \equal {#1} {} }
%    { \def\temp {#2} }  % if #1 == blank
%    { \def\temp {#1} }   % else (not blank)
%    \refstepcounter{todocounter}\todo[color=jacolor,inline,caption={\textbf{\thetodocounter. JA} \temp}]{\textbf{\thetodocounter. JA:} #2}{}}
%\newcommand{\jablock}[2][{}]{
%    \ifthenelse { \equal {#1} {} }
%    { \def\templist {\emph{block comment}}
%        \def\tempheader {}}  % if #1 == blank
%    { \def\templist {#1}
%        \def\tempheader {#1}}   % else (not blank)
%    \refstepcounter{todocounter}\todo[color=jacolor,inline,caption={\textbf{\thetodocounter. JA} \templist}]{\textbf{\thetodocounter. JA: \tempheader}\\\begin{minipage}{\textwidth}#2\end{minipage}}{}}
%\newcommand{\jafigure}[1]{\refstepcounter{todocounter}\missingfigure[figcolor=jacolor]{\textbf{\thetodocounter. JA:} #1}{}}
%
%
%\colorlet{dpcolor}{yellow!40!white}
%\newcommand{\dpnote}[1]{\refstepcounter{todocounter}\todo[color=dpcolor,linecolor=black,size=\scriptsize,caption={\textbf{\thetodocounter. DP} #1}]{\textbf{\thetodocounter. DP:}\\#1}{}}
%\newcommand{\dpinline}[2][]{
%    \ifthenelse { \equal {#1} {} }
%    { \def\temp {#2} }  % if #1 == blank
%    { \def\temp {#1} }   % else (not blank)
%    \refstepcounter{todocounter}\todo[color=dpcolor,inline,caption={\textbf{\thetodocounter. DP} \temp}]{\textbf{\thetodocounter. DP:} #2}{}}
%\newcommand{\dpblock}[2][{}]{
%    \ifthenelse { \equal {#1} {} }
%    { \def\templist {\emph{block comment}}
%        \def\tempheader {}}  % if #1 == blank
%    { \def\templist {#1}
%        \def\tempheader {#1}}   % else (not blank)
%    \refstepcounter{todocounter}\todo[color=dpcolor,inline,caption={\textbf{\thetodocounter. DP} \templist}]{\textbf{\thetodocounter. DP: \tempheader}\\\begin{minipage}{\textwidth}#2\end{minipage}}{}}
%\newcommand{\dpfigure}[1]{\refstepcounter{todocounter}\missingfigure[figcolor=dpcolor]{\textbf{\thetodocounter. DP:} #1}{}}
%
%\colorlet{cpcolor}{red!40!white}
%\newcommand{\cpnote}[1]{\refstepcounter{todocounter}\todo[color=cpcolor,linecolor=black,size=\scriptsize,caption={\textbf{\thetodocounter. CP} #1}]{\textbf{\thetodocounter. CP:}\\#1}{}}
%\newcommand{\cpinline}[2][]{
%    \ifthenelse { \equal {#1} {} }
%    { \def\temp {#2} }  % if #1 == blank
%    { \def\temp {#1} }   % else (not blank)
%    \refstepcounter{todocounter}\todo[color=cpcolor,inline,caption={\textbf{\thetodocounter. CP} \temp}]{\textbf{\thetodocounter. CP:} #2}{}}
%\newcommand{\cpblock}[2][{}]{
%    \ifthenelse { \equal {#1} {} }
%    { \def\templist {\emph{block comment}}
%        \def\tempheader {}}  % if #1 == blank
%    { \def\templist {#1}
%        \def\tempheader {#1}}   % else (not blank)
%    \refstepcounter{todocounter}\todo[color=cpcolor,inline,caption={\textbf{\thetodocounter. CP} \templist}]{\textbf{\thetodocounter. CP: \tempheader}\\\begin{minipage}{\textwidth}#2\end{minipage}}{}}
%\newcommand{\cpfigure}[1]{\refstepcounter{todocounter}\missingfigure[figcolor=cpcolor]{\textbf{\thetodocounter. CP:} #1}{}}
%
%\colorlet{hlcolor}{purple!40!white}
%\newcommand{\hlnote}[1]{\refstepcounter{todocounter}\todo[color=hlcolor,linecolor=black,size=\scriptsize,caption={\textbf{\thetodocounter. HL} #1}]{\textbf{\thetodocounter. HL:}\\#1}{}}
%\newcommand{\hlinline}[2][]{
%    \ifthenelse { \equal {#1} {} }
%    { \def\temp {#2} }  % if #1 == blank
%    { \def\temp {#1} }   % else (not blank)
%    \refstepcounter{todocounter}\todo[color=hlcolor,inline,caption={\textbf{\thetodocounter. HL} \temp}]{\textbf{\thetodocounter. HL:} #2}{}}
%\newcommand{\hlblock}[2][{}]{
%    \ifthenelse { \equal {#1} {} }
%    { \def\templist {\emph{block comment}}
%        \def\tempheader {}}  % if #1 == blank
%    { \def\templist {#1}
%        \def\tempheader {#1}}   % else (not blank)
%    \refstepcounter{todocounter}\todo[color=hlcolor,inline,caption={\textbf{\thetodocounter. HL} \templist}]{\textbf{\thetodocounter. HL: \tempheader}\\\begin{minipage}{\textwidth}#2\end{minipage}}{}}
%\newcommand{\hlfigure}[1]{\refstepcounter{todocounter}\missingfigure[figcolor=hlcolor]{\textbf{\thetodocounter. HL:} #1}{}}
%
%
%\colorlet{focolor}{orange!40!white}
%\newcommand{\fonote}[1]{\refstepcounter{todocounter}\todo[color=focolor,linecolor=black,size=\scriptsize,caption={\textbf{\thetodocounter. FO} #1}]{\textbf{\thetodocounter. FO:}\\#1}{}}
%\newcommand{\foinline}[2][]{
%    \ifthenelse { \equal {#1} {} }
%    { \def\temp {#2} }  % if #1 == blank
%    { \def\temp {#1} }   % else (not blank)
%    \refstepcounter{todocounter}\todo[color=focolor,inline,caption={\textbf{\thetodocounter. FO} \temp}]{\textbf{\thetodocounter. FO:} #2}{}}
%\newcommand{\foblock}[2][{}]{
%    \ifthenelse { \equal {#1} {} }
%    { \def\templist {\emph{block comment}}
%        \def\tempheader {}}  % if #1 == blank
%    { \def\templist {#1}
%        \def\tempheader {#1}}   % else (not blank)
%    \refstepcounter{todocounter}\todo[color=focolor,inline,caption={\textbf{\thetodocounter. FO} \templist}]{\textbf{\thetodocounter. FO: \tempheader}\\\begin{minipage}{\textwidth}#2\end{minipage}}{}}
%\newcommand{\fofigure}[1]{\refstepcounter{todocounter}\missingfigure[figcolor=focolor]{\textbf{\thetodocounter. FO:} #1}{}}
%
%\colorlet{occolor}{purple!40!white}
%\newcommand{\ocnote}[1]{
%    \refstepcounter{todocounter}
%    \todo[color=occolor,
%    linecolor=black,
%    size=\scriptsize,
%    caption={\textbf{\thetodocounter. OC} #1}
%    ]{\textbf{\thetodocounter. OC:}\\#1}{}
%}
%\newcommand{\ocinline}[2][]{
%    \ifthenelse { \equal {#1} {} }
%    { \def\temp {#2} }  % if #1 == blank
%    { \def\temp {#1} }   % else (not blank)
%    \refstepcounter{todocounter}
%    \todo[color=occolor,
%    inline,
%    caption={\textbf{\thetodocounter. OC} \temp}
%    ]{\textbf{\thetodocounter. OC:} #2}{}
%}
%\newcommand{\ocblock}[2][{}]{
%    \ifthenelse { \equal {#1} {} }
%    { \def\templist {\emph{block comment}}
%        \def\tempheader {}}  % if #1 == blank
%    { \def\templist {#1}
%        \def\tempheader {#1}}   % else (not blank)
%    \refstepcounter{todocounter}\todo[color=occolor,inline,caption={\textbf{\thetodocounter. OC} \templist}]{\textbf{\thetodocounter. OC: \tempheader}\\\begin{minipage}{\textwidth}#2\end{minipage}}{}}
%\newcommand{\ocfigure}[1]{
%    \refstepcounter{todocounter}
%    \missingfigure[figcolor=occolor]{\textbf{\thetodocounter. OC:} #1}{}
%}

